\documentclass{article}
\usepackage[utf8]{inputenc}
\usepackage{fancyhdr} % Required for custom headers
%\usepackage{lastpage} % Required to determine the last page for the footer
\usepackage{extramarks} % Required for headers and footers
\usepackage[usenames,dvipsnames]{color} % Required for custom colors
\usepackage{graphicx} % Required to insert images
\usepackage{listings} % Required for insertion of code
\usepackage{courier} % Required for the courier font
\usepackage{lipsum} % Used for inserting dummy 'Lorem ipsum' text into the template
\usepackage{enumerate}
\usepackage{multicol}
\usepackage{caption}
\usepackage{subcaption}
\usepackage{ulem} % underline emph
\usepackage{amsmath} % for \text in mathmode
\usepackage[hypcap]{caption}

% Margins
\topmargin=-0.45in
\evensidemargin=0in
\oddsidemargin=0.5in
\textwidth=5.5in
\textheight=9.0in
\headsep=0.25in

\linespread{1.3} % Line spacing

% Set up the header and footer
\pagestyle{fancy}
\lhead{} % Top left header
\chead{\hmwkClass: \hmwkTitle} % Top center head
\rhead{\firstxmark} % Top right header
\lfoot{\lastxmark} % Bottom left footer
\cfoot{\thepage} % Bottom center footer
%\rfoot{Page\ \thepage\ of\ \protect\pageref{LastPage}} % Bottom right footer
\renewcommand\headrulewidth{0.4pt} % Size of the header rule
\renewcommand\footrulewidth{0.4pt} % Size of the footer rule

\setlength\parindent{0pt} % Removes all indentation from paragraphs

\definecolor{MyDarkGreen}{rgb}{0.0,0.4,0.0} % This is the color used for comments
\lstloadlanguages{Matlab} % Load C syntax for listings, for a list of other languages supported see: ftp://ftp.tex.ac.uk/tex-archive/macros/latex/contrib/listings/listings.pdf
\lstset{language=C, % Use python in this example
        frame=single, % Single frame around code
        basicstyle=\small\ttfamily, % Use small true type font
        keywordstyle=[1]\color{Blue}\bf, % C functions bold and blue
        keywordstyle=[2]\color{Purple}, % C function arguments purple
        keywordstyle=[3]\color{Blue}, % Custom functions \underbar underlined and blue
        identifierstyle=, % Nothing special about identifiers
        commentstyle=\usefont{T1}{pcr}{m}{sl}\color{MyDarkGreen}\small, % Comments small dark green courier font
        stringstyle=\color{Purple}, % Strings are purple
        showstringspaces=false, % Don't put marks in string spaces
        tabsize=5, % 5 spaces per tab
        %
        % Put standard Python functions not included in the default language here
        morekeywords={rand},
        %
        % Put Python function parameters here
        morekeywords=[2]{on, off, interp},
        %
        % Put user defined functions here
        morekeywords=[3]{glutCreateWindow,p},
       	%
        morecomment=[l][\color{Blue}]{...}, % Line continuation (...) like blue comment
        numbers=none, % can use none % Line numbers on left
        firstnumber=1, % Line numbers start with line 1
        numberstyle=\tiny\color{Blue}, % Line numbers are blue and small
        stepnumber=1 % Line numbers go in steps of 5
}
% \usepackage{graphicx}
\newcommand{\indep}{\rotatebox[origin=c]{90}{$\models$}}

% Creates a new command to include a perl script, the first parameter is the filename of the script (without .pl), the second parameter is the caption
\newcommand{\code}[1]{
\begin{itemize}
\item[]\lstinputlisting[label=#1]{#1.c}
%\item[]\lstinputlisting[caption=#2,label=#1]{#1.c}
\end{itemize}
}

%----------------------------------------------------------------------------------------
%	DOCUMENT STRUCTURE COMMANDS
%	Skip this unless you know what you're doing
%----------------------------------------------------------------------------------------

\setcounter{secnumdepth}{0} % Removes default section numbers

\newcommand{\homeworkProblemName}{}
\newenvironment{homeworkProblem}[1]{ % Makes a new environment called homeworkProblem which takes 1 argument (custom name) but the default is "Problem #"
    \renewcommand{\homeworkProblemName}{#1} % Assign \homeworkProblemName the name of the problem
    \section{\homeworkProblemName} % Make a section in the document with the custom problem count
}

\newcommand{\problemAnswer}[1]{ % Defines the problem answer command with the content as the only argument
    \noindent\framebox[\columnwidth][c]{\begin{minipage}{0.98\columnwidth}#1\end{minipage}} % Makes the box around the problem answer and puts the content inside
}

\newcommand{\homeworkSectionName}{}
\newenvironment{homeworkSection}[1]{ % New environment for sections within homework problems, takes 1 argument - the name of the section
    \renewcommand{\homeworkSectionName}{#1} % Assign \homeworkSectionName to the name of the section from the environment argument
    \subsection{\homeworkSectionName} % Make a subsection with the custom name of the subsection
}

%----------------------------------------------------------------------------------------
%	NAME AND CLASS SECTION
%----------------------------------------------------------------------------------------

\newcommand{\hmwkTitle}{Project 1, PI Approximation } % Assignment title
\newcommand{\hmwkDueDate}{\date{March 08, 2017}} % Due date
\newcommand{\hmwkClass}{TMA4280\\ Introduction to Supercomputing\\} % Course/class
\newcommand{\hmwkAuthorName}{Neshat\ Naderi}  % Your name


%----------------------------------------------------------------------------------------
%	TITLE PAGE
%----------------------------------------------------------------------------------------

\title{
\vspace{2in}
\textmd{\textbf{\hmwkClass:\ \hmwkTitle}}\\
\normalsize\vspace{0.1in}\normalsize{\hmwkDueDate}
\vspace{0.1in}\large{\text{Compiler Construction}}
\vspace{3in}
}

\author{\textbf{\hmwkAuthorName}}
\date{} % Insert date here if you want it to appear below your name

%----------------------------------------------------------------------------------------
\begin{document}
\maketitle

% \setcounter{tocdepth}{1} % Uncomment this line if you don't want subsections listed in the ToC

% \newpage
% \tableofcontents
%\newpage

%----------------------------------------------------------------------------------------
%	PROBLEM 1
%----------------------------------------------------------------------------------------

% To have just one problem per page, simply put a \clearpage after each problem
\clearpage

\begin{homeworkProblem}{}
  \begin{homeworkSection}{Unit test}
    Unit tests could be useful for verifying if the results expected from a program is returned correctly. So in order to check the output result from a parallel program we could look into the errors of calculation within a parallel program comparing to a serial one. Because parallel programs could return different values of floating points operation because of the rounding in different stages like partial computation by processes. While in a serial program it is done at the end of the computation.
  \end{homeworkSection}

  \section{Verification test}
  Results for $k = 1,..,24$ shows that the error for Reimann Zeta function decreases by increasing the size of $k$. The convergence for Machin formula happens very early in computation and it doese not change even comparing the summation to \pi value with more than $100$ decimals. So the approximation of Machin formula is more accurate compared to Reimann Zeta.

  \section{Data Distribution}
  Load balancing of the work between processes could be callenging when the computation time could vary for each vector elements. In addition if the length of the vector is not dividable by the number of processses the root process could be loaded by more work than others. Like in this implementation the root takes care of both data distribution and I/O in addition to the computational operations. That could prevent the perfect parallel computation performance such that processors wait busy until the root processsors is completed the distribution. In this exercise the length is assumed to be power of $2$.

  \section{MPI implementation}
  ADD PLOTS, ERROR AND TIMING\\~\\
  Parallel program using MPI is compiled and executed on a WHICH MACHINE with \# number of processors.\\
  MPI calls were first of all necessary for environment initialisation and termination as listed below:
  \begin {lstlisting}
    MPI_Init(&argc, &argv);
    MPI_Comm_rank(MPI_COMM_WORLD, &rank);
    MPI_Comm_size(MPI_COMM_WORLD, &nproc);
    MPI_Finalize();

  \end {lstlisting}






\end{homeworkProblem}


\end{document}
